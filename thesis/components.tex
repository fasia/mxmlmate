%### 
\subsection{System Components}
%### 
\label{sec:components}
The extension of \xmlmate to binary subjects benefits from frameworks described in \cref{sec:tech} as they 
help solve the majority of the arising subtasks rather easily. Let me first give you a superficial description
of the new extended \xmlmate work process. 
\begin{enumerate}
  \item \xmlmate performs its genetic operations on a population of chromosomes which results in a set of 
  \xml files, whose fitness (according to the currently employed fitness function) needs to be determined.
  The paths to these files are then sent out via a \zmq \texttt{PUSH} socket either directly to the 
  \emph{test drivers}, in case the program being tested supports reading inputs in \xml format, or alternatively 
  to an arbitrary number of \emph{converters}, that, as the name suggests, convert the \xml files into a format 
  suitable for the system under test. This distinction is completely transparent to \xmlmate and thus allows for 
  adding an arbitrary amount of transformation steps between itself and the test drivers.
  \item A converter (most of which are implemented in \python) receives a task message via a 
  \zmq \texttt{PULL} socket, which it completes by converting the file found at the location specified in the message. 
  It then sends out another message with the location of the converted file via its \texttt{PUSH} socket further 
  down the processing pipeline to the \emph{broker}.
  There can be an unlimited number of converters active at the same time, but there is only one broker.
  \item The broker is responsible for collecting conversion results from converters and fairly distributing them 
  to the test drivers. The broker could be very efficiently implemented in \python as the fair 
  queueing forwarding functionality is available out of the box with \zmq.
  \info{It would be cool to have actual load balancing brokers between \xmlmate and converters and between 
  converters and test drivers, but it's somewhat tricky, so it remains a cool future work item}
  \item A test driver receives a message either directly from \xmlmate in case of an \xml based file format, 
  or otherwise from the broker, unpacks the file path, and feeds it to the system under test, which is being 
  monitored by a \emph{pintool} that implements the ``client side" of the aforementioned currently 
  employed fitness function. The test driver's responsibilities also include signaling the beginning and end 
  of the \texttt{sut} processing to the pintool by calling special marker methods \texttt{PIN\_SCORE\_START} and
  \texttt{PIN\_SCORE\_END}, which the pintool replaces with its own internal fitness data related processing methods. 
  The driver is implemented in {\small C} and, as with the converter, there can be arbitrarily many active at the same time.
  \item As previously mentioned, the pintool monitors the execution of the program under test and records
  data relevant to the computation of the fitness score according to the fitness function it is part of.
  E.g. for a fitness function that counts the number of basic blocks executed in the program, the pintool 
  would keep a set of basic blocks that it has observed being executed. The pintool replaces the call to 
  \texttt{PIN\_SCORE\_START} in the driver with a method that resets its fitness related data stores (in the 
  above example it would empty the set of basic blocks). It also replaces the call to \texttt{PIN\_SCORE\_END} 
  with a method that sends out the stored data out to \xmlmate via \zmq. The pintools are implemented in \cpp 
  and there is always exactly one pintool per test driver.
  \item The fitness function in \xmlmate receives the message from the pintool, interprets it according to 
  its specification (e.g. again, if the fitness function is supposed to count the number of executed basic blocks, 
  it would expect to receive a set of basic block addresses) and finally assigns the computed fitness score to 
  the genetic representation of the \xml file sent out in step 1.
%   \item liveness guard
\end{enumerate}
\improvement[inline]{Add illustration}