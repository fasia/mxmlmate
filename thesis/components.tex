%### 
\subsection{System Components}
%### 
\label{sec:components}
The extension of \xmlmate to binary subjects benefits from frameworks described in \cref{sec:tech} as they 
help solve the majority of the arising subtasks rather easily. Let me give you a superficial description
of the components involved in the new extended \xmlmate work process. 
\begin{itemize}
  \item The \java application \xmlmate performs its genetic operations on a population of chromosomes; this 
  results in a set of \xml files, whose fitness needs to be determined according to the currently employed
  fitness function. The paths to these files are then sent out via a \zmq socket either directly to the
  \emph{test drivers}, in case the program being tested supports reading inputs in \xml format, or alternatively 
  to an arbitrary number of \emph{converters}, that, as the name suggests, convert the \xml files into a format 
  suitable for the system under test. This distinction is completely transparent to \xmlmate and thus allows for 
  adding an arbitrary number of transformation steps between itself and the test drivers.
  
  \item A load balancer is responsible for managing a set of homogeneous worker nodes like converters or test
  drivers by distributing the arriving workloads fairly among them. The fair queueing provides a significant
  performance upgrade from the previously employed round robin method. Not all workloads require equal
  processing time, and because it is indeterminable a priori, the round robin workload distribution strategy
  has often caused faster work items to get ``stuck'' in queue behind slower ones, while there were idle
  workers available. The new load balancing mechanism allows to track the worker's availability and assign
  newly arriving workloads to idle workers in the least recently used order. To provide an analogy, the load
  balancer's queueing strategy has been upgraded from a supermarket to a post office. 
  
  A load balancer provides a facade for the workers it manages, so that other system components only ever
  directly interact with the balancer, but never with the workers themselves, as they are sometimes unreliable
  and can fail at any moment. The most common system setup consists of two load balancers - one for the format
  converters and one for test drivers, so even though it says above that \xmlmate sends its work packets to
  converters or test drivers, in actuality it communicates with their respective load balancers. 
    
  \item A format converter (most of which are currently implemented in \python) receives a conversion task from
  its load balancer through a \zmq socket, which it completes by converting the file found at the location
  specified in the task message. It then responds with another message containing the location of the converted
  file. There can be an unlimited number of converters active at the same time processing multiple
  conversion requests in parallel. 
  
  The design decision to enable the converters to drop in and out at
  runtime has two very useful properties: firstly, it allows to circumvent the parallelism limitation
  imposed by {\small Python's} GIL (global interpreter lock) by being able to run multiple interpreter
  instances in parallel without them interfering; and, secondly, it makes it possible to apply meta-heuristics
  to the process by varying the number of converters dynamically depending on the load, which makes for an
  interesting future work item. \unsure{Mention in future work section.}
  
  \item A test driver receives a message from its balancer, which, in turn, receives it either directly from
  \xmlmate in case of an \xml based file format, or otherwise from the previous balancer (most likely a 
  converter's balancer), unpacks the file path, and feeds it to the system under test, which is being monitored
  by a \emph{pintool} that implements the ``client side" of the aforementioned currently employed fitness
  function. The test driver's responsibilities also include signaling the beginning and end of the execution of
  the program under test to the pintool by calling special marker methods \texttt{PIN\_SCORE\_START}
  and \texttt{PIN\_SCORE\_END}, which the pintool replaces with its own internal fitness data related
  processing methods. The driver is implemented in {\small C} and, as is the case with converters, there can
  be arbitrarily many active at the same time.
  
  \item As previously mentioned, the pintool monitors the execution of the program under test and records
  data relevant to the computation of the fitness score according to the fitness function it is part of.
  E.g. for a fitness function that counts the number of basic blocks executed in the program, the pintool 
  would keep a set of basic blocks that it has observed being executed. The pintool replaces the call to 
  \texttt{PIN\_SCORE\_START} in the driver with a method that resets its fitness related data stores (in the 
  above example it would clear the set of executed basic blocks). It also replaces the call to
  \texttt{PIN\_SCORE\_END} with a method that sends out the stored data back to the balancer. 
  The pintools are implemented in \cpp and there is always exactly one pintool per test driver.
  Because the pintool runs in the same process as the test driver, it is possible to share the same \zmq
  socket between the driver and pintool for communication with the load balancer, which makes it relatively
  easy to implement the following system component.
  
  \item A lifeguard protects the joint process of test driver, system under test and pintool from an untimely
  death. Whenever \xmlmate produces an input file which crashes the program under test, the entire process dies
  off without a possibility to report this valuable finding. To prevent this, a lifeguard launches a driver
  process and gives it an \emph{identity} (which is a \zmq concept - the load balancer distinguishes the
  workers by their identities). As soon as control returns to the lifeguard, which means that the launched
  process has died, it assumes the identity of the recently deceased and reports the fact of death to the load
  balancer, which will transparently forward this to \xmlmate in order for the corresponding input to be
  stored. Afterwards, the lifeguard casts off the identity and relaunches the driver with it once again.
  Each test driver is being guarded by its own personal lifeguard implemented in \python.
  
  \item The fitness function in \xmlmate receives the message from the pintool, interprets it according to 
  its specification (e.g. again, if the fitness function is supposed to count the number of executed basic blocks, 
  it would expect to receive a set of basic block addresses) and finally assigns the computed fitness score to 
  the genetic representation of the \xml file sent out in step 1. Unless the message says that the
  corresponding input has caused a test driver to die, in which case the genetic representation shall receive
  the worst fitness score to be removed from the population in order not to continuously keep crashing the test
  drivers, and the input file itself shall be kept permanently.
\end{itemize}
\improvement[inline]{Add illustration of system components}

\unsure[inline]{Maybe describe mockups used in the development process}