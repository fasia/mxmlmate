%###
\subsection{Challenges}
%###
Getting away from pure \java and adding support for application binaries as test subjects 
to \xmlmate by transforming it from 
a single \java application into a set of loosely coupled, yet interdependent and distinct 
processes written in different programming languages presented a number of challenges from 
such fields like software engineering and system architecture design. 
In the following sections I would like to shed some light onto 
the most noteworthy of them.
%###
\subsubsection{Chromosome Representation}
%###
\label{sec:repr}
\xmlmate is implemented on top of the genetic algorithm framework that is part of 
\evosuite\cite{6004309} and as such must adhere to its specifications and structures 
of genetic representations. 
Because \evosuite as a whole is aimed at generating {\small JUnit} tests for \java applications, 
the entities that comprise the population  (i.e. the \emph{chromosomes} or \emph{individuals}) 
are called \emph{test suites}. The test suites are what evolves from generation to generation.
In turn, the test suites consist of \emph{tests}. Similarly, \xmlmate has adopted the concepts of 
test suite and test they correspond to a set of \xml trees and a single \xml tree, respectively. 
They are sometimes referred to as \emph{test suite chromosomes} and \emph{test chromosomes}, resp.


In \xmlmate all \xml trees are governed by a single \xsd that is given as a parameter at program start.
In order to implement the genetic operation \emph{mutation} for test chromosomes more or less 
efficiently each node in a test's \xml tree should have a reference to its corresponding definition 
in the schema. This is accomplished by extending some of the \java classes provided by \xom to include the 
references in question as well as methods that maintain their relevance and correctness. This alone, 
however, is not sufficient for implementing an effective \emph{crossover} of two \xml trees.

To quickly find sites suitable for crossover in two test chromosomes an additional mapping is needed: 
one that maps definitions in the global schema to nodes that are actually present in the \xml tree, so 
as to be able to find compatible intersections.
Since I am using \xerces for storing and accessing the schema information, and it is only exposed via 
a \java interface, it was not possible to add this mapping to local copies of the schema  
enhanced with the needed references to test chromosomes. However, what could be added were local 
maps referencing the global definition instances as keys and local nodes as values. A disadvantage 
is that additional methods had to be added to ensure the relevance of these mappings, 
as the node composition of an \xml tree is constantly changing. 
One benefit of this approach is the reduced memory consumption as the schema
definition must only be stored once globally, and the local mappings are nothing but sets of references 
internally and thus very lightweight.
\improvement[inline]{Add a picture of chromosome representation}
%### 
\subsubsection{Communication Protocol}
%###
Since \xmlmate now comprises several processes, a communication protocol must be put in place to 
interconnect the individual components. The communication channels themselves are built on top of
the \zmq messaging library with \msgpack as the serialization/deserialization engine. The current
implementation uses tcp loopback\footnote{TCP loopback is used because the pure \java implementation 
of \zmq does not support \texttt{ipc} on the Windows platform, which I do most of the development on.
Using a JNI based Java wrapper as a drop-in replacement would alleviate this problem, 
but it is not considered to be worth the effort.} 
as a carrier for the messages; however, it is entirely possible
to change this to inter-process communication just by changing a configuration option, should it 
become necessary. As yet, it is very far from being a bottleneck.


After each evolution step a test suite usually contains several tests that changed due to mutation and
crossover. When the fitness of the suite is requested to be computed next time by the genetic algorithm, 
only the tests that changed should be evaluated. For each of those a file is written and a packet is 
formed consisting of a number identifying the test inside the test suite and the path to its corresponding 
file. This packet is serialized into a byte string with \msgpack and then sent as a \zmq message via a 
\texttt{PUSH} socket on a well known port (e.g. the default is 5556). After all messages have been sent,
a response is awaited for each of those sent out tasks. Responses are collected as \zmq messages coming 
in via a \texttt{PULL} socket on another well known port (default 5557) and deserialized with \msgpack 
into the identifying number and whatever data format is expected by the current fitness function. 
(E.g. a fitness function that counts the number of executed basic blocks expects an array of addresses.)
As far as the \java part of \xmlmate is concerned this is all it ever sees of the entire process, and so it 
must rely on the pintool monitoring the application under test to collect and send back fitness results.


A format converter receives a \zmq message via \texttt{PULL} on port 5556, unpacks it with \msgpack, 
stores the identifying number, converts the file, packs the stored number followed by the path to 
the converted file with \msgpack into another \zmq message and sends it out via \texttt{PUSH} on port 5560.


The broker is very simple: it pulls in messages on port 5560 and pushes them out again unaltered on port 5570.


A test driver listens on port 5570 (or 5556 if there are no converters) for incoming \zmq messages and unpacks
them with \msgpack into the identifying number and a path. It then calls \texttt{PIN\_SCORE\_START()} to 
signal the start of the execution of the system under test to the monitoring pintool, whereafter it passes the 
received file path to the sut for execution. When the sut finishes and control returns to the test driver, 
it calls \texttt{PIN\_SCORE\_END(number)} with the number stored from the message earlier.


When a test driver calls \texttt{PIN\_SCORE\_START()}, the call gets replaced with a call to a method in the 
pintool that clears its internal information buffers, thereby resetting all currently gathered fitness related 
information. When the test driver then calls \texttt{PIN\_SCORE\_END(number)}, this call gets replaced with 
one that packs and sends out a message containing the identifying number and any fitness information via \zmq 
\texttt{PUSH} on port 5557.

Please note that all ports mentioned above are customizable and even the message carrier can be changed from tcp 
to any other protocol supported by \zmq, of which there are plenty.
\improvement{Mention \zmq binding and connecting and static and dynamic system components}
\unsure{mention starting synchronization}

\change[inline]{make pretty communication graph}
%###
\subsubsection{Format Converters}
%###
One of the main tasks of this work is to show that \xmlmate is versatile enough to not be limited to 
only \xml and \xml-derived formats. In order to facilitate this versatility the concept of a format 
converter has been introduced. The main goal of a converter for a specific format is to convert an 
\xml file corresponding to a schema specification into another format as specified by the purpose 
of this converter. For an example consider a converter from \xml to the packet capture format \pcap. 
This converter expects \xml files in the format corresponding to the \pcap{} \xsd and produces a 
valid \pcap file. \info{make a section reference}


\xmlmate also has support for starting out its evolution process from a given set of inputs. If the   
inputs are not available in the \xml format, they must first be converted so as to be genetically 
representable. For this use case there are reverse converters, which convert files into their 
\xml equivalent, again, according to the corresponding \xsd. For the above example a reverse converter 
could transform any valid \pcap file into its \xml representation governed by the \pcap{} \xsd.
%###
\subsubsection{Parallelization}
%###
After running the first tests on the initial \xmlmate design, which did not include any parallelization, 
it quickly became clear that both format conversion and running the program under test with \pin 
are two performance bottlenecks significantly slowing down the entire process. Considering that the 
tests were run on a multicore machine, and only one CPU was ever doing any work, the decision was 
made to make both the conversion and evaluation steps parallel by allowing arbitrarily many 
converter and test driver instances (along with their corresponding pintools) to participate in the process. 
This was somewhat easily accomplished due to the very well thought out design of the messaging library \zmq, 
as there were only minor issues to consider - like adding the static broker component to 
manage the dynamic number of converters and test drivers.

However, after adding the aforementioned parallelization, a new bottleneck presented itself - this time 
somewhere inside the \java part of \xmlmate. Using a \java profiling tool I was able to trace the cause 
to the mutation and crossover routines, which seems logical in retrospect because both manipulate the 
in-memory representations of \xml tree structures. That alone, however, is not sufficient to cause a 
major bottleneck. \evosuite's genetic algorithm always makes backups of entire test suites as well as 
single tests before applying any genetic operation to them in order to be able to recover from an 
unfavorable outcome (e.g. decreased fitness) and roll back all changes. So each time an in-memory \xml 
tree was going to be mutated, it was deeply copied, each time two \xml trees were going to be crossed 
over, they were both deeply copied, each time an entire test suite was going to be mutated, it was 
deeply copied - each of its tests one after another, and each time two test suites were about to 
undergo crossover, both of them were deeply copied. That was what caused this particular bottleneck. 
To remedy this situation I applied the following two improvements:

The first one was adding three distinct task executors dividing the responsibilities of deeply copying, mutating and 
crossing over \xml trees. This way all three classes of tasks were enabled to run simultaneously 
no longer needing to queue up linearly one strictly after another. The mutation task executor is 
responsible for taking a single individual \xml tree, copying it and mutating the copy; the 
crossover task executor takes two \xml trees, delegates their copying to the copying task executor 
for a speed up, and then performs the actual crossover on the returned copies.

The second improvement was making the process of copying the test suites shallow instead of deep. 
This is a safe change since the individual tests are still copied before any changes are applied 
to them - the result is a kind of \emph{copy-on-maybe-write} policy as only those tests are copied 
that are going to experience potential change. \info{reference this in local search}

After that there were still other bottlenecks - some more surprising than others. For example 
it was rather sudden to see calls to the \texttt{Random} class being listed among the most time consuming 
by the profiler. In \java 7 the \texttt{Random} class is thread-safe, which has lead to a lot of 
contention when multiple threads suddenly started asking the singleton instance of \evosuite's randomness 
provider for random values for mutation and crossover. The fix was rather easy: \java 7 also provides 
a \texttt{ThreadLocalRandom} class intended for exactly this kind of situation; this easy fix came 
at a price, however - it is no longer possible to provide a single seed value to the whole generation 
process.

Another minor performance optimization included adding a polymorphic method for selecting a random 
element from a given \texttt{Set}. The previously available methods treated a \texttt{Set} like a 
\java \texttt{Collection} and performed the selection by first copying the entire set into a 
\texttt{RandomAccess}-able \texttt{List}. This is clearly wasteful of resources in case the set in question is large.
In this case it actually faster to generate a random number between zero and the number of elements in 
the set and then actuate the set's iterator that number of times finally claiming the element acquired 
last as the required random choice.

One further possible major optimization could be to allow partial fitness score evaluation while a suite 
mutation is still in progress. When an \xml file has left the mutation or crossover executor it could be 
immediately sent out to worker processes for fitness evaluation. Preliminary tests showed promising 
results; however, this optimization is very intrusive of \evosuite's concept of a single threaded 
genetic algorithm possibly leading to grave incompatibilities with other mechanisms deep within other parts 
of \evosuite, therefore it was deemed as a neat future work item, which should probably be combined with a 
proper restructuring of the genetic framework.
%###
\subsubsection{Local Search}
%###
As previously mentioned, one of the core principles of \xmlmate was to always create schema-valid inputs; 
however, this no longer complies with current requirements in the use case of security testing. 
Fortunately, it is possible to use another artifact of \evosuite's to provide specification deviations: 
the local search. 