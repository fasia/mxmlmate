\section{Related Work}
\label{sec:relwork}
There is an entire class of approaches which put an emphasis on the diversity of the produced inputs and the
efficiency of their generation over guidance towards any certain criterion. These are collectively known as
\emph{fuzzers}, typically their primary goal is to find failures and vulnerabilities, and they can be
categorized as follows.
\subsection{Blackbox Fuzzers}
Blackbox fuzzers have no knowledge of the application they are testing whatsoever -- they simply generate random
inputs that can be fed into the program under test to see if it can handle the input or if it crashes.
There are many tools that can be at least partially classified as blackbox fuzzers. For \xml-based formats
there are the integrated development environments Eclipse\cite{eclipse} and Visual Studio\cite{visual} which
are both capable of generating simple instances for a given \xsd. The oXygen\cite{oxygen} tool is a
dedicated stand-alone \xml instance generator, which makes it, in contrast to the previous examples, a true
fuzzer.
 		 
The TAXI\cite{Bertolino:2007:ATD:1270230.1270257} approach presents a variation on the same principle, but
offers an improvement by making use of partition testing: the number of instances to be generated
and tested can be significantly reduced under the assumption that the ranges of values specified in the \xsd
can be divided into partitions, such that if the application under test can accept and handle one input from a
partition, it can do so with all the other values from the same partition.

Wuzhi Xu et al.\cite{1544740} have applied blackbox fuzzing on web services and extended the approach by
performing perturbations on the \xsd before generating instances in order to additionally test the resilience
of the validation mechanism.

Codenomicon Labs have developed a proprietary XML fuzzing framework\cite{codenomicon} as part of their
\textsc{cross} initiative (Codenomicon Robust Open Source Software). According to their own description their
fuzzer is capable of introducing various kinds of malformities into the generated inputs -- breaking the
encoding, repetition of tags and elements, dropping of tags and elements, recursive structures, overflows,
special characters, and many others.

\subsection{Whitebox Fuzzers}
In contrast to blackbox fuzzers, the whitebox fuzzers have some knowledge of the application under
test, which they use to get some degree of feedback on the inputs they generate. They mostly use this feedback
to refrain from generating duplicates, or inputs which induce the same behavior in the tested program, thus
avoiding useless work effort and making testing more efficient.

Named after the rabbit breed ``American Fuzzy Lop", \emph{afl-fuzz}\cite{afl} is a whitebox fuzzer
primarily aimed at testing applications that take a binary file as input. Not unlike \xmlmate, it uses a
combination of genetic algorithm and lightweight dynamic instrumentation. Its main focus is on
efficiency, which means that while it can generate inputs for simple formats like \texttt{JPEG} reasonably
well\footnote{\url{http://lcamtuf.blogspot.de/2014/11/pulling-jpegs-out-of-thin-air.html}}, it fails
to produce more sophisticated ones like \png, which is riddled with internal consistency checksums.

The SAGE\cite{godefroid-sage} tool employs dynamic symbolic analysis of x86 binary code of the program under
test to obtain information about the inputs to be generated: while the program is executing, constraints on
the input format are gathered, whereafter a constraint solver is used to generate new inputs. SAGE requires
existing files as a seed because path explosion and shortcomings of symbolic execution can make starting with
empty inputs infeasible.

\subsection{Language-Specific Fuzzers}
As the probability to generate valid inputs that penetrate deep into the application logic without considerable
format specification efforts is usually infinitesimally small, it is sometimes preferable to constrain the
range of generated inputs by means of a grammar for a specific language.

LangFuzz\cite{holler2012} is a typical blackbox fuzzer that leverages this principle and offers a possibility
to provide input fragments, which it will then attempt to make use of in the process of input generation.

In their later work \cite{Godefroid:2008:GWF:1375581.1375607} the authors of SAGE have extended the approach to
take a grammar for input generation; however, their extension does not benefit from any existing inputs for
grammar-based test generation seeding.

\subsection{Search-Based Testing}
In the category of search-based approaches, i.e.\ those that employ some form of guidance criteria to steer the
generation of test data in the desired direction, there are approaches conceptually close to \xmlmate.

\evosuite{}\cite{fraser2013whole} is a dedicated unit test generator aimed at \java programs. By
dynamically observing the execution of the application under test and aiming for maximum code coverage, it
utilizes a genetic algorithm to generate entire JUnit test suites. It provides multiple code coverage
criteria as search goals, which makes it a very versatile tool. In fact, the implementation of \xmlmate is
in part based on \evosuite.

\textsc{Exsyst}\cite{gross-issta2012} is also based on \evosuite, but it specializes in testing of \java
GUI applications by generating series of interactions that aim to explore and test as many of the program's
features as possible. The result is not a test suite, but rather a set of executable GUI interactions that
provide considerable code coverage, or even reveal failures. In contrast to pure \evosuite, this approach
performs testing on the system level by exercising the tested application via the same interface as its
intended users, this means that any failure found by \textsc{Exsyst} is a real failure and not a false
positive. \xmlmate, also being a system level tool, has this property as well.