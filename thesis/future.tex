\section{Future Work}
\label{sec:future}
The presented work paves the way for many more useful improvements and extensions on numerous facets and
aspects of \xmlmate. This section gives a brief overview of some of the most interesting future work items.

%###
\paragraph{Better Parallelization} ~\\
%###
One particularly large work item is completely rewriting the
genetic algorithm of \xmlmate from scratch aiming for better parallelization support from the beginning. At the time of writing it is estimated that this
goal is best achieved by making heavy use of many of the features that the {\small
Scala}\cite{scala-overview-tech-report} programming language has to offer, as well as utilizing the {\small
Akka} actor-based message-driven concurrency framework\cite{Wyatt:2013:AC:2663429} to make \xmlmate fully
concurrent and distributed.
%###
\paragraph{Meta-Heuristics} ~\\
%###
To make better use of the available computational resources, the genetic algorithm at the heart of
\xmlmate can profit from the addition of meta-heuristics that would govern the number of instances of certain
subprocesses like format converters or worker nodes during runtime based on the proportion of their
respective load on the system.
%###
\paragraph{Specification Violation} ~\\
%###
Delving deeper into the use case of finding vulnerabilities, it might be wise to further investigate
the mechanism responsible for generating aberrant and unexpected input values and extend it to work not only
with integer and floating point types, but other types and even structures permissible in \xml as well.
%###
\paragraph{Further Fitness Functions} ~\\
%###
Since \xmlmate provides an interface for pluggable fitness functions, it is only natural to want to try
out more of them. Some next candidates would be the branch coverage fitness function as implemented in
\evosuite in the context of generating test suites, and for program resilience testing purposes a fitness
function that aims to maximize the number of context switches in a concurrent program seems interesting for
finding multi-threading related issues such as deadlocks, livelocks, starvation and the like.
%###
\paragraph{Hardware Counters and Virtualization} ~\\
%###
For future fitness functions that use program performance values in the computation of their
fitness score it is possible to significantly increase efficiency by making use of hardware performance
counters, which are provided by modern hardware and can be directly accessed in most operating systems by
means of special APIs like the ones provided by the {\small PAPI}\cite{Mucci99papi:a} or {\small LTTng}\cite{
combined_tracing_ols2009} libraries. Note that this would impose a limitation on the concurrent use of \xmlmate
on a single machine; however, this limitation can probably be solved by using virtualization techniques like
{\small Docker}\cite{docker} and {\small Vagrant}\cite{vagrant} -- especially since \xmlmate is now a
distributed system capable of running across multiple distinct hardware nodes. This arrangement can also be of
special benefit to creating a fitness function specialized in detecting buffer overflows by means of advanced
operating system features like memory page marking and eviction. One library specializing in such techniques
of memory debugging is {\small DUMA}\footnote{\url{http://sourceforge.net/projects/duma/}}, which is a
spiritual heir to the discontinued Electric Fence project.
%###
\paragraph{More Efficient Message Transport} ~\\
%###
If \xmlmate is going to be used in truly distributed systems, it is probably a good idea to replace
the currently employed \msgpack message serialization and deserialization engine by another, more efficient and
sophisticated library. When real network traffic is involved, it seems like a good idea to minimize the amount
of traffic that has to pass through the network, and having a library with a streaming interface that also
allows to deserialize messages before they have reached their destination completely is certainly also of
good use as well. Some candidate replacements can be found in \cref{sec:msgpack}.
%###
\paragraph{Robust Communication} ~\\
%###
Further elaborating the idea of distributed execution environments in the presence of unreliable
nodes, it seems wise to invest some effort into extending the inter-component communication protocol for
reliability and message delivery guarantees.
%###
\paragraph{Additional Test Subjects} ~\\
%###
With more fitness functions come more test subjects that can be explored. For formats, for which the
corresponding schemas and converters already exist in the current version, such applications as \emph{tcpdump}
and \emph{tshark} for the \pcap format and a whole
lot\footnote{\url{http://www.libpng.org/pub/png/pngapcv.html}} of programs for \png are available.
Adding test subjects that process other file formats requires adding new schemas and converters that
support them. Some envisioned targets are \emph{libjpeg}, \emph{giflib} and \emph{libflac} libraries for
processing image, animation and sound files, respectively.
%###
\paragraph{Result Minimization} ~\\
%###
In terms of usability and convenience \xmlmate could profit from minimizing the files responsible for
revealing a defect in the application under test using delta debugging\cite{zeller2002simplifying} to make it
easier for the user to find the actual cause of the failure in question.
%###
\paragraph{Crossing the Programming Language Barrier} ~\\
%###
Ending on another big topic, seeing as \xmlmate has its origins in the \java world, it could be extended
to support testing of \java applications that use native code by performing instrumentation on both \java
byte code and the native binary code. 

{\color{white}{Thank you for reading the digital version and not printing this out on paper! :)}}