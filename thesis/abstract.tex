\section*{Abstract}
% \addcontentsline{toc}{section}{Abstract}
\change{Rewrite abstract}
%Motivation 
The e{\bfseries X}tensible {\bfseries M}arkup {\bfseries L}anguage (XML) is 
a universal data interchange format for a wide variety of applications. 
One of the challenges that comprise automated testing of such applications is the generation of meaningful inputs that 
can advance deep into the program flow; 
%Problem
however, this is a hard task for most format-agnostic test generation tools. 
Because XML files are complexly structured, simple guessing techniques are insufficient for producing valid and realistic inputs. 
One can leverage XML schemas, which describe the structure of XML files, for generating valid inputs, and 
using a guided search technique can further increase their effectiveness. 
Also, unit-level testing tools are usually unaware of certain preconditions that hold when 
the tested application is run normally, thus they produce unrealistic tests that expose false positives as they 
violate those preconditions. System-level testing, however, does not have this problem because system tests follow the 
natural execution order of the tested application. 
%Methodology
This thesis presents an approach for performing automated search-based system testing of 
applications that process XML inputs that are governed by XML schemas. 
I have implemented a prototype, applied it to three java applications, and compared its performance with 
that of a state of the art search-based unit testing tool. 
%Results
By utilizing search-based system-level testing in conjunction with the structural information from 
XML schemas my approach was able to generate meaningful and realistic tests, which have uncovered real defects in the 
tested applications.  