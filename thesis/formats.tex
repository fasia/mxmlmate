%### 
\subsection{Formats and Subjects}
%###
\xmlmate now supports producing inputs in almost any format, provided there is an \xsd and a converter
available. Let me describe some formats that were experimented with along with their schemas, converters, and
subject programs in the next sections.
%### 
\subsubsection{XML}
%###
Before I proceed with describing other exciting formats, I want to emphasize that by being extended to an all
new multi-component distributed system \xmlmate has by no means lost its ability to interact with subject
programs that actually process inputs in the \xml format. As a matter of fact, \xml as such is not exactly a
full-fledged format in and of itself, it is rather a meta format, which many other formats are subsumed by. One
such example is the ubiquitous \texttt{html} format, which, coincidentally, I have chosen as the basis format
for testing \texttt{libxml} - a library for processing generic \xml documents written in \texttt{C}, which is
\unsure{add examples, e.g. python's lxml}
widely used as the foundation of an entire multitude of parser implementations. 
\texttt{html} has a very complex specification, which exhibits almost all of \xml's feature set.
%### 
\subsubsection{PNG}
%###
 
%### 
\subsubsection{Pcap}
%###
 
%### 
\subsubsection{Flac}
%###
 