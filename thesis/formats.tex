%### 
\subsection{Formats and Subjects}
%###
\xmlmate now supports producing inputs in almost any format, provided there is an \xsd and a converter
available. Let me describe some formats that were experimented with along with their schemas, converters, and
subject programs in the next sections.
%### 
\subsubsection{XML}
%###
Before I proceed with describing other exciting formats, I want to emphasize that by being extended to an all
new multi-component distributed system \xmlmate has by no means lost its ability to interact with subject
programs that actually process inputs in the \xml format. As a matter of fact, \xml as such is not exactly a
full-fledged format in and of itself, it is rather a meta format, which many other formats are subsumed by. One
such example is the ubiquitous \texttt{html} format, which, coincidentally, I have chosen as the basis format
for testing \texttt{libxml} - a library for processing generic \xml documents written in \texttt{C}, which is
\unsure{add examples, e.g. python's lxml}
widely used as the foundation of an entire multitude of parser implementations. 
\texttt{html} has a very complex specification, which exhibits almost all of \xml's feature set.
%### 
\subsubsection{PNG}
%###
 
%### 
\subsubsection{Pcap}
%###
\label{sec:formats:pcap}
\begin{listing}[H]
\centering
\begin{tabular}{|m{4.5cm}m{8cm}|}
\begin{xmlcode}
<pcap xmlns="http://www.xmlmate.org/pcap">
	<header>
		<version_major>2</version_major>
		<version_minor>4</version_minor>
		<thiszone>0</thiszone>
		<sigfigs>40</sigfigs>
		<snaplen>80</snaplen>
		<network>1</network>
	</header>
	<packet>
		<ts_sec>49</ts_sec>
		<ts_usec>4</ts_usec>
		<incl_len>80</incl_len>
	</packet>
	<packet>
		<ts_sec>64</ts_sec>
		<ts_usec>9</ts_usec>
		<incl_len>80</incl_len>
	</packet>
</pcap>
\end{xmlcode}
&
\begin{tikzpicture}[grow=right]
\tikzset{level distance=68pt,sibling distance=0pt}
\tikzset{execute at begin node=\strut}
	\Tree [.pcap 
[.packet [.incl\_len 80 ] [.ts\_usec 9 ] [.ts\_sec 64 ] ]
[.packet [.incl\_len 80 ] [.ts\_usec 4 ] [.ts\_sec 49 ] ]
[.header [.network 1 ] [.snaplen 80 ] [.sigfigs 40 ] [.thiszone 0 ] [.version\_minor 4 ] [.version\_major 2 ] ]
]
\end{tikzpicture}
\end{tabular}
\caption{Example \xml for \texttt{pcap} as Text and Tree}
\label{lst:xmlexample}
\end{listing}
% This listing can probably be made as long as needed to increase the number of pages by adding packets ;)

\begin{listing}[H]
\centering
\inputminted[frame=lines,fontsize=\small]{xml}{../subjects/pcap/schema/pcap.xsd}
\caption{\xsd for the \texttt{pcap} File Format}
\end{listing}

%### 
% \subsubsection{Flac}
%###
 