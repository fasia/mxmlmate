%### 
\subsection{File Formats}
\label{sec:formats}
%###
\xmlmate now supports producing inputs in almost any format, provided there is an \xsd and a converter
available. Let me describe some formats that were experimented with along with their schemas, converters, and
subject programs in the next sections.
%### 
\tocless\subsubsection{xhtml}
%###
Before I proceed with describing other exciting formats, I want to emphasize that by being extended to an all
new multi-component distributed system \xmlmate has by no means lost its ability to interact with subject
programs that actually process inputs in the \xml format. Technically, \xml as such is not exactly a
full-fledged format, it is rather a meta format, which many other formats are subsumed by. One
such example is the E\textbf{x}tensible \textbf{H}ypertext \textbf{M}arkup \textbf{L}anguage format
\texttt{xhtml}, which was created in an effort to force strict consistency rules onto web pages designed for
rendering on \emph{thin clients} -- devices with weak computational resources, for which it would be infeasible
to implement graceful handling of inconsistent web page code (e.g.\ mismatched or missing tags, etc.). 
 
Because \texttt{xhtml} is in essence a simple \xml wrapper around \texttt{html}, it is possible to use standard
and computationally lightweight \xml parsers instead of specialized lenient \texttt{html} parsers. It is
further possible to use \xml validation techniques to discard inconsistent web pages without wasting resources trying
to render them.

The main differences of \texttt{xhtml} from \texttt{html} include the mandatory presence of the
\texttt{<html>}, \texttt{<head>}, \texttt{<title>} and \texttt{<body>} elements, as well as the {\small
<!DOCTYPE \ldots>} document type declaration, in addition to the \texttt{xmlns} default namespace declaration
attribute on the root \texttt{<html>} element, of which there must be exactly one. Furthermore, all element
tags must be properly nested and always closed. Element and attribute names must be lowercase, and the
latter's values must always be quoted.

The World Wide Web Consortium provides an official \xsd for
\texttt{xhtml}\footnote{http://www.w3.org/TR/xhtml1-schema/} which is used in the \xmlmate experiments
described in \cref{sec:evaluation}. With its 2211 lines this schema is too long to be included
in this document.

%### 
\tocless\subsubsection{pcap}
%###
\label{sec:formats:pcap}
The \textbf{p}acket \textbf{cap}ture file format is rather simple in that it has no deeply nested structures or
overly strict rules on placement and integrity of its contents. \Cref{lst:pcapfile} shows the layout of a
typical \pcap file -- it consists of a global header containing various global information followed by records
for each captured packet, consisting of a packet header and data.

\begin{figure}[htb]
\centering
\begin{bytefield}[boxformatting={\centering\tiny}]{32}
\colorbitbox{mygreen}{8}{Global Header}{tbr} &
\colorbitbox{myorange}{8}{Packet Header}{tbr} &
\colorbitbox{myorange}{8}{Packet Data}{tbr} &
\colorbitbox{myblue}{8}{Packet Header}{tbr} &
\colorbitbox{myblue}{8}{Packet Data}{tbr} &
\bitbox[tb]{2}{\ldots}
\end{bytefield}
\caption{Pcap File Structure}
\label{lst:pcapfile}
\end{figure}

Technical specifications of the individual headers are presented in \cref{lst:pcapformat}. 
Briefly, the global header consists of the following elements:
\texttt{magic\_number} is hexadecimal \texttt{a1b2c3d4} for determining the endianness of the file,
\texttt{version\_major} and \texttt{version\_minor} denote the format version (currently 2.4),
\texttt{thiszone} is the correction time in seconds between GMT and the current timezone (this value is
usually set to 0); \texttt{sigfigs} is the accuracy of time stamps (also usually set to 0), \texttt{snaplen} is
the snapshot length -- i.e.\ the maximum number of bytes per captured packet; and finally \texttt{network}
specifies link-layer header type of the captured packets (e.g.\ Ethernet, 802.11, PPP, etc.).

The individual packet headers are also very simple and structured as follows:
\texttt{ts\_sec} and \texttt{ts\_usec} provide timing information about when the packet was captured,
\texttt{incl\_len} is the number of bytes of data actually included in the capture file (possibly
limited by \texttt{snaplen} from the global header), while \texttt{orig\_len} is the
real length of the packet as it appeared on the network.
The packet data immediately follows the packet header as a blob of \texttt{incl\_len} bytes.


\begin{listing}[htb]
\centering
\begin{cppcode}
typedef struct pcap_hdr_s {
        guint32 magic_number;   /* magic number */
        guint16 version_major;  /* major version number */
        guint16 version_minor;  /* minor version number */
        gint32  thiszone;       /* GMT to local correction */
        guint32 sigfigs;        /* accuracy of timestamps */
        guint32 snaplen;        /* max length of captured packets, in octets */
        guint32 network;        /* data link type */
} pcap_hdr_t;

typedef struct pcaprec_hdr_s {
        guint32 ts_sec;         /* timestamp seconds */
        guint32 ts_usec;        /* timestamp microseconds */
        guint32 incl_len;       /* number of octets of packet saved in file */
        guint32 orig_len;       /* actual length of packet */
} pcaprec_hdr_t;
\end{cppcode}
\caption{Pcap File Format}
\label{lst:pcapformat}
\end{listing}

This format seems to be rather straightforward to write a corresponding \xsd for. \Cref{lst:xsdexample} shows
this schema. The root element of the \xml representation of \pcap is the \texttt{pcap} element, which has the
type PCAPType. This type defines its contents as a \texttt{header} element of the type GlobalHeaderType
followed by zero or more \texttt{packet} elements of the type PacketHeaderType. 

Do not be confused by the ``tns:'' prefix found throughout the schema -- it is just a marker assigning the
declared entities to the target namespace ``\url{http://www.xmlmate.org/pcap}'', this is a mechanism for
disambiguating identically named declarations across multiple schema definitions. 

The GlobalHeaderType is almost identical in its contents to the format definition given in
\cref{lst:pcapformat} with the exception of the missing declaration of \texttt{magic\_number}, which will be
automatically inserted by the corresponding format converter because the endianness of the final \pcap file
will depend entirely on the one used by the converter anyway.

The PacketHeaderType similarly leaves the \texttt{orig\_len} value to be filled in by the converter because, as
per format definition, it must not be smaller than \texttt{incl\_len}, which is generated randomly when
instantiating the \xml. The converter additionally ensures that the value of \texttt{incl\_len} does not exceed
the one of \texttt{snaplen} from the global header, which is impossible to do by means of the schema
definition alone.

The schema further lacks the definition for the data blob that represents the actual packet data itself, once
again it is the converter's job to provide that according to the (now possibly trimmed) \texttt{incl\_len}
value. Besides, it would be somewhat wasteful to first generate a blob of data in string form only to 
convert it to a binary format.

\begin{listing}[htp]
\centering
\inputminted[frame=lines,fontsize=\small]{xml}{../subjects/pcap/schema/pcap.xsd}
\caption{\xsd for the \texttt{pcap} File Format}
\label{lst:xsdexample}
\end{listing}

The above shows pretty well that the schema definition and the code of the corresponding converter
must be well coordinated. It also shows that a considerable amount of functionality can be placed into the
converter component, although this is not necessary as the design of \xmlmate allows to plug transformer
entities of arbitrary nature into the processing pipeline, so it might be favorable to have separate
components responsible for correcting inconsistent values between \xmlmate and the format converters to achieve
a greater separation of concerns -- in the end it's all about trade-offs.

The schema also introduces the type \texttt{positiveInt} as \texttt{unsignedInt} with a minimum value of 1
because there is no corresponding type among the \xsd built-ins.

\Cref{lst:xmlexample} shows a possible instantiation of the described schema -- the left side depicts the
textual representation that gets sent out to the converter, and the right side represents the instance as an
in-memory \xml tree as operated upon by \xmlmate internally.


\begin{listing}[H]
\centering
\begin{tabular}{|m{4.5cm}m{8cm}|}
\begin{xmlcode}
<pcap xmlns="http://www.xmlmate.org/pcap">
	<header>
		<version_major>2</version_major>
		<version_minor>4</version_minor>
		<thiszone>0</thiszone>
		<sigfigs>40</sigfigs>
		<snaplen>80</snaplen>
		<network>1</network>
	</header>
	<packet>
		<ts_sec>49</ts_sec>
		<ts_usec>4</ts_usec>
		<incl_len>80</incl_len>
	</packet>
	<packet>
		<ts_sec>64</ts_sec>
		<ts_usec>9</ts_usec>
		<incl_len>80</incl_len>
	</packet>
</pcap>
\end{xmlcode}
&
\begin{tikzpicture}[grow=right]
\tikzset{level distance=68pt,sibling distance=0pt}
\tikzset{execute at begin node=\strut}
	\Tree [.pcap 
[.packet [.incl\_len 80 ] [.ts\_usec 9 ] [.ts\_sec 64 ] ]
[.packet [.incl\_len 80 ] [.ts\_usec 4 ] [.ts\_sec 49 ] ]
[.header [.network 1 ] [.snaplen 80 ] [.sigfigs 40 ] [.thiszone 0 ] [.version\_minor 4 ] [.version\_major 2 ] ]
]
\end{tikzpicture}
\end{tabular}
\caption{Example \xml for \texttt{pcap} as Text and Tree}
\label{lst:xmlexample}
\end{listing}
% This listing can probably be made as long as needed to increase the number of pages by adding packets ;)

%### 
\tocless\subsubsection{PNG}
%###
The \textbf{P}ortable \textbf{N}etwork \textbf{G}raphics format is, as its name suggests, a graphics file
format often abbreviated \png. It was mainly designed for sharing of images over the Internet,
which is why it also supports lossless data compression and is riddled with internal consistency checks by
means of CRC-32 checksums. Compared to the previously presented \pcap format, \png is very complex; it bears,
however, some similarity to \pcap in that \png files are also divided into smaller entities called
\emph{chunks}. There are many kinds of chunks defined in the \png standard, but \cref{lst:pngfile} shows only
the most basic ones. 

\begin{figure}[H]
\centering
\begin{bytefield}[bitheight=\widthof{~{\tiny MAGIC}~},boxformatting={\centering}]{32}
\bitbox{1}{\rotatebox{90}{{\tiny MAGIC}}} &
\bitbox{6}{IHDR} &
\bitbox[tbr]{6}{PLTE} &
\bitbox[tb]{8}{} &
\bitbox[tb]{4}{IDAT} &
\bitbox[]{3}{\ldots} &
\bitbox[tbr]{5}{} &
\bitbox[tbr]{6}{IEND}
\end{bytefield}
\caption{Simple Example of PNG File Structure}
\label{lst:pngfile}
\end{figure}

A \png file always starts with the magic byte sequence \texttt{89 50 4e 47 0d 0a 1a 0a}, which, in contrast to
\pcap, is not part of the first chunk. All chunks have a common structure as shown in \cref{lst:pngchunks},
whereby the CRC-32 checksum is computed over the chunk type and chunk data, but not the length. The chunk type
is identified with four ASCII letters and the case of each letter has a specific meaning for the decoder.
The chunk is recognized as \emph{critical} if the first letter is uppercase and \emph{ancillary} otherwise.
Critical chunks must be understood by the decoder program as they contain information indispensable for
rendering the file, if a decoder does not recognize a critical chunk, it is required to abort the processing or
supply an appropriate warning. Ancillary chunks contain information optional to overall processing of the file.

% The meaning of the case of other letters is not considered important enough to take up space in this document
% - if you wish to actually learn the format, it is far more advisable to visit an appropriate specification
% page\footnote{For example \url{http://www.w3.org/TR/PNG/}}.

\begin{figure}[H]
\centering
\begin{bytefield}[boxformatting={\centering\small}]{32}
\colorbitbox{lightgray}{7}{Length}{tlr} &
\colorbitbox{lightgray}{7}{Chunk Type}{tlr} &
\colorbitbox{lightgray}{18}{Chunk Data}{tlr} &
\colorbitbox{lightgray}{7}{CRC-32}{tlr} \\
\bitbox{7}{4 bytes} &
\bitbox{7}{4 bytes} &
\bitbox{18}{\emph{Length} bytes} &
\bitbox{7}{4 bytes}
\end{bytefield}
\caption{PNG Chunk Structure}
\label{lst:pngchunks}
\end{figure}

To give you an impression as to the nature of some chunks defined in the format specification, the following
listing offers a small selection of chunk types and their descriptions.
\begin{description}
	\item[\texttt{IHDR}] is the header chunk, it must be the first one in the \png file. It contains important
	global information about the image: its width, height, bit depth, color type, as well as compression, filter,
	and interlace methods.
	\item[\texttt{PLTE}] contains a list of colors also known as a palette. The palette contains up to 256 color
	entries, each of which consists of three bytes denoting values of RGB, respectively.
	\item[\texttt{IDAT}] carries the data of the image filtered and compressed according to the specification
	found in the \texttt{IHDR} chunk of this image. There can be several contiguous \texttt{IDAT} chunks sharing
	the image data payload, which is especially useful when loading images over a slow Internet connection, so
	that rendering can be started while the image is still partially in transit.
	\item[\texttt{IEND}] marks the end of the image.
	\item[\texttt{tRNS}] conveys transparency information for datastreams that do not include a so called alpha
	channel.
	\item[\texttt{gAMA}] specifies gamma correction for the given image file.
\end{description}

There are many more chunk types, but unfortunately they cannot be placed randomly all over a \png file as the
documentation specifies certain rules regarding their usage. This causes a lot of complexity to writing an
\xsd to reflect the \png format properly. Consider the following rules:

\begin{itemize}
\item[] The \emph{color type} in the \texttt{IHDR} chunk can have values of 0, 2, 3, 4, and 6, which place
specific restrictions on several other elements e.g.\ the allowed bit depth value, as well as the presence of
the \texttt{PLTE} chunk.

\item[] The \texttt{PLTE} chunk must be present with the color type set to 3, it is optional with color types 2
and 6, and it is illegal with the types 0 and 4.

\item[] If the \texttt{PLTE} chunk appears, it must do so before any \texttt{IDAT} chunk.

\item[] The only chunks allowed to appear multiple times are \texttt{IDAT}, \texttt{sPLT}, \texttt{iTXt},
\texttt{tEXt}, and \texttt{zTXt}.

\item[] The \texttt{cHRM}, \texttt{gAMA}, \texttt{iCCP}, \texttt{sBIT}, and \texttt{sRGB} chunks must appear
before the \texttt{PLTE} and \texttt{IDAT} chunks.

\item[] The \texttt{bKGD}, \texttt{hIST}, and \texttt{tRNS} chunks must appear after the \texttt{PLTE} chunk,
but before any \texttt{IDAT} chunk.
\end{itemize}

These are only some of the constraints defined in the \png format. Unfortunately the mechanisms provided by
the \xsd language are insufficient to express all the constrains easily. However, there is a possibility to
capture partial ordering constraints and restrict legal value combinations by ``exploding'' the schema. This
means that every combination is declared as its own element and then all of the possible applications of the
decision as to which combination to instantiate are expressed by means of a \texttt{choice} particle.
For an example consider \cref{lst:schemaexplosion}, which depicts the different legal combinations of color
type and bit depth expressed as an exploded \texttt{choice} particle. Each listed element declaration includes
a slightly different \texttt{IHDR} subtype with values for color type and bit depth set accordingly.
Furthermore, the presence of a \texttt{PLTE} chunk is governed by this \texttt{choice} because it depends on
the color type as mentioned in the rules above. These explosions have lead the schema document to grow to 3106
lines in size distributed over 21 files. More complicated format requirements like checksum correctness and conformity of
the width and height of an image and the amount of data in \texttt{IDAT} chunks had to be moved into the format
converter.

\begin{listing}[H]
\centering
\begin{cminted}[frame=lines,tabsize=2,fontsize=\small]{xml}
<choice>
	<element ref="tns:ChunksCT0BD1" />
	<element ref="tns:ChunksCT0BD2" />
	<element ref="tns:ChunksCT0BD4" />
	<element ref="tns:ChunksCT0BD8" />
	<element ref="tns:ChunksCT0BD16"/>
	<element ref="tns:ChunksCT2BD8" />
	<element ref="tns:ChunksCT2BD16"/>
	<element ref="tns:ChunksCT3BD1" />
	<element ref="tns:ChunksCT3BD2" />
	<element ref="tns:ChunksCT3BD4" />
	<element ref="tns:ChunksCT3BD8" />
	<element ref="tns:ChunksCT4BD8" />
	<element ref="tns:ChunksCT4BD16"/>
	<element ref="tns:ChunksCT6BD8" />
	<element ref="tns:ChunksCT6BD16"/>
</choice>
\end{cminted}
\caption{XML Schema Explosion Fragment}
\label{lst:schemaexplosion}
\end{listing}

%### 
% \subsubsection{Flac}
%###
