\section{Evaluation}
\label{sec:evaluation}
\subsection{Subjects}
% todo add sources for test subjects
\subsubsection{libxml2}
\texttt{libxml2} is a library for processing generic \xml documents, which is written in
{\small C}. It is widely used and provides bindings to many programming languages including \cpp, 
{\small C\#}, \python{}, {\small Ruby}, {\small PHP5}, and {\small Perl}. The popular \texttt{libxslt} library
for processing \xml stylesheet transformations is also largely based on \texttt{libxml2}.
\subsubsection{libpcap}
\subsubsection{libpng}
% mention generating only 2x2 pixels
%\subsubsection{libflac}
\subsection{Results}
\subsection{Threats to Validity}
As any empirical study, this evaluation is subject to several threats to validity.
First, as regards \emph{external validity}, the subjects used in this study may be far from representative of
the entirety of all applications now compatible with \xmlmate, and may even be uncharacteristic among their
respective application class.

Concerning \emph{internal validity} it must be kept in mind that search-based testing is random in nature, and
thus the results may vary greatly over multiple experiments. Some effort was made to mitigate this problem by
performing multiple runs and taking the average results. However, it is unclear if a much larger sample size
would alter the results in any significant way.

Furthermore, the choice of many of the parameters for the genetic algorithm such as population size,
genetic operation probabilities, or chromosome limitations might not have been optimal for the chosen test
subjects.

Regarding \emph{construct validity} it is uncertain if the metrics used in the evaluation are well suited
to adequately measure the usefulness of the presented approach.
 
% partial bias -> pcap schema and converter was written by me, I tried to adhere to the specs