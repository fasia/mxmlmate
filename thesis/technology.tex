%### 
\subsection{Technology Stack}
%### 
\label{sec:tech}
Over the time of its development \xmlmate has amassed quite a technology stack, 
which has grown to include not only a number of \xml processing libraries, but also communication, instrumentation, and serialization frameworks. 
In the following sections I would like to give a brief overview of the technologies used.
%###  
\subsubsection{Xom}
%### 
\xom\cite{xom} is a \java library for handling \xml documents at the core of \xmlmate. 
It offers in-memory representation of \xml tree structures as well as support for Namespaces in \xml, {\small XPath 1.0}, {\small XSLT 1.0}, 
{\small XInclude}, {\small xml:id}, {\small xml:base}, Canonical \xml, and Exclusive Canonical \xml.
I decided for it in favor of {\small SAX}, {\small StAX}, {\small DOM4j} and {\small jDOM} because of its simplicity and efficiency.
It is also the only \xml API that ensures correctness very strictly - \xom only allows to create namespace well-formed XML documents, 
which coincides with one of the underlying principles of \xmlmate{} - generating valid and well-formed data.
Furthermore, \xom is open to extension, which allows for easy enhancements and adaptations to make it suitable 
for building the basis of genetic representations of \xml trees (further described in \cref{sec:repr}).
\improvement[inline]{Add some xml specification and example}
%### 
\subsubsection{Xerces2}
%### 
\xerces\cite{xerces} is a \java library for parsing, validating and manipulating XML documents, which most importantly to \xmlmate, has support for W3C XML Schema 1.1 (Working Drafts, December 2009). 
At the heart of \xmlmate lies a representation of an \xsd, which is used as a blueprint for generating new \xml instances and modifying existing ones. 
This representation is implemented with \xerces, as at the time of conception it was really the only available \xsd implementation for \java. 
\xerces mainly provides access to the definitions of a schema as well as a mechanism for validating \xml instances.
Unfortunately, it only exposes its functionality via \java interfaces, which makes enhancements and adaptations rather hard, but not impossible, as you will see in \cref{sec:repr}.
\improvement[inline]{Add some xsd specs and example}
%### 
\subsubsection{PIN}
%### 
{\small Intel} \pin\cite{pin} is a dynamic binary instrumentation framework for the IA-32 and x86-64 instruction-set architectures 
that enables the creation of dynamic program analysis tools, which performs the instrumentation at run time on  
compiled binary files. Thus, it requires no recompiling of source code and can support instrumenting programs that dynamically generate code.
\pin allows a tool to insert arbitrary code (written in {\small C} or \cpp) in arbitrary places in the executable, for which it 
provides an API that abstracts away the underlying instruction-set idiosyncrasies and allows context information 
such as register contents to be passed to the injected code as parameters. It also automatically saves and restores 
the registers that are overwritten by the injected code so the application continues to work.

Generally, the instrumentation with \pin consists of two components: a mechanism that decides where and what code to insert, 
and the code to be executed at insertion points. These two components, called \emph{instrumentation} and \emph{analysis} 
code are hosted in a single executable called a \emph{Pintool}, which functions like a plugin to the \pin framework.
A pintool can register callbacks on different levels of granularity, varying from single instructions to procedures
to entire binary images, in order to receive context-dependent information and access to values of particular interest.
\unsure[inline]{maybe add small example}
%### 
\subsubsection{ZeroMQ}
%### 
{\small ZeroMQ}\cite{zmq} (or \zmq as it as actually called) is an extremely efficient messaging library, 
which is very well suited for use in concurrent or distributed applications. \zmq regards messages as 
completely transparent blobs of data which are to be transported across predefined communication channels 
between sockets. Rather than unnecessarily defining its own transfer protocol, \zmq works on top of already exiting 
ones like \texttt{inproc}, \texttt{IPC}, \texttt{TCP}, \texttt{TIPC} and multicast such as \texttt{pgm} or \texttt{epgm}.
Out of the box \zmq provides its users with several communication patterns that can be either used directly or combined 
into more complex patterns, which remain easy to manage and use. Some basic patterns are \texttt{Request/Response}, 
\texttt{Publish/Subscribe}, or \texttt{Push/Pull} among others.
There are implementations of \zmq in many programming languages, of which the ones for \java, \python, {\small C}, 
and \cpp are actually used in \xmlmate.
\unsure{Add something more - maybe the fact that it uses the same code over all protocols}
\unsure{mention RabbitMQ and Nanomsg}
%### 
\subsubsection{MessagePack}
%### 
Because a messaging library only solves the problem of getting \emph{arbitrary data} between peers, it 
alone does not suffice for creating a fully fledged communication protocol - this is where serialization/deserialization libraries 
usually come in. \msgpack\cite{msgpack} is one such serialization format that is highly efficient as it based on 
binary representation of data. It is implemented as a library in at least 20 programming languages. 
Once again, \xmlmate uses the ones for \java, \python, {\small C}, and \cpp.
While \msgpack is very efficient in what it does, it has some disadvantages such as 
\begin{itemize}
  \item Integer values are limited to be in $[-2^{63}, 2^{64}-1]$.
  \item The maximum length of an array or string is limited to $2^{32}-1$.
  \item It is the user's responsibility to ensure correct endianness across all endpoints.
\end{itemize}

There is another relatively young binary serialization format \emph{CBOR}\footnote{\url{http://tools.ietf.org/html/rfc7049}} 
that does not have many of {\small MessagePack's} disadvantages, while being comparatively as efficient.
There are implementations in {\small C}, \python and \java; however, at the time of {\small XMLMate's}
conception I did not know of their existence. This might be a subject of some future enhancement.
\info{Mention in Future Work}
% C (https://github.com/upwhere/ccbor), 
% Python (https://code.google.com/p/cbor/) 
% and Java (https://github.com/c-rack/cbor-java)
\unsure{Mention Protobuf, Cap'n Proto, SBE, and FlatBuffers}
% optimizations in caching results (e.g. Trove)